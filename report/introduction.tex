\section{Introduction}
\label{sec:introduction}

Broadband Internet services are not widely available in rural areas such as Mendocino
County in California. Why not?

One answer is that, due to the spare populations in these areas, it is expensive to deploy traditional network technologies such as fiber.
As a result, there is often only a single incumbent Internet service provider serving these areas, stifling competition and driving up prices.
Recently, however, organizations such as Further Reach have started to offer competitive Internet services, helping to lower prices for subscribers.

Another possibility might be that the people of Mendocino have concerns about the prospect of pervasive broadband services. It is this possibility that we seek to investigate.
Do some residents harbor concerns about broadband, or even oppose it? Do some
residents not believe that broadband would be beneficial for their lives? If so,
what are their concerns, and how common are these concerns?

To answer these
questions, in this project we conduct a qualitative research study on
perceptions of broadband services in Mendocino county. The first and main part
of this paper (sections \S\ref{sec:background}, \S\ref{sec:methods} \& \S\ref{sec:results}) summarizes
our findings from nine interviews with Mendocino residents.

As technologists, we generally believe that access to broadband Internet is a valuable
good. If some residents wish to know more about broadband Internet technologies, or harbor concerns that might be easily addressed through
education, we would like to address them if possible. The second part of this
paper (starting at \S\ref{sec:education-materials}) summarizes our efforts to develop educational materials with this goal
in mind. We designed
these materials based on an analysis of our interview materials: for each category of
easily addressable concerns voiced about broadband Internet, we gathered
factual information and presented them in an easy-to-understand format.

Throughout this process, we learned a great deal! We end the paper (section
\S\ref{sec:lessons}) by summarizing these lessons.

%%% Local Variables:
%%% mode: latex
%%% TeX-master: "main"
%%% End:
