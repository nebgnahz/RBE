\section{Results}
\label{sec:results}

We first present the individual results from each method we discussed in
\autoref{subsec:qualitative_research}. Then we summarize and synthesize the
final results.

\subsection{Interviews with FR}
\label{sec:interviews-with-fr}

In order to understand some of the common perceptions people have regarding
broadband in the Mendocino County, we decided to have an interview session
with Yahel and Zean due to their on-site communication expertise and their
valuable experience which they have collected over the years while setting up
Further Reach. Our initial goal was to determine some of the common
misconceptions of people regarding broadband, which they came across while
promoting Further Reach. We also wanted to gain insight into some of the
common topics people have expressed interest in learning more about.

A few important aspects that we touched upon during the interview process were as follows:
\begin{itemize}
\item Some of the common questions asked by the users and the mode of communication used by them to express their concerns
\item Some of the common technical problems that have occurred in the network
since it was set up
\item The average education level of the local population regarding the notion of broadband and the usage of the Internet in general
\item The techniques used by Further Reach in order to help people understand how the setup of the network works
\item Types of technical problems encountered by the users in the past and the level of detail they used to describe them
\item Queries of the users that have been difficult to solve or the users themselves had trouble understanding the explanation given to them for their concerns
\item The specific components of the network that require regular maintenance by the local technicians
\end{itemize}

From the interviews, it became evident that there were indeed some concerns
and misconceptions that the local population had. While interviewing Zean, he
informed us that some users are confused with the term `MB'. Specifically,
they seem to conflate `Bandwidth' with `Bandwidth caps'. Some users also have
very little knowledge about network latency and the amount of Internet
bandwidth required for different types of services such as VOIP, video streaming, email etc. He also emphasized the need to explain to the people the difference between satellite and wireless Internet connections.

From the interview session with Yahel, we came to know that some people have
concerns regarding the effects of radiation from wireless networks. These
concerns are one of the main reasons that prevent people from signing up for
high-speed Internet services. He also told us that it has been a regular
practice for them to need to explain people about Line of Sight (LOS) and what causes its obstruction.

An important observation that both of them made was that most of the people
within the Further Reach network are quite aware of the importance of
broadband and the various services and usages it comes with. Moreover, Zean
does a pretty thorough job of answering people's questions as he sets up their
home network.

Thus the education material that we would prepare should aim towards the
population who do not have access to high-speed Internet access nor any ways
to gain information about it. As Zean puts it, ``we don't have a store
front for people to go to get their questions answered''.

\subsection{Support Tickets Analysis}
\label{sec:supp-tick-analys}

In order to reduce the scope of our research problem, we first decided to identify some of the common types of concerns and queries that people have when they have already been provided with high-speed Internet connection. For this, we used the ticketing system used by Further Reach’s team for tracking its customer’s questions. We analyzed over 1425 tickets filed since August 2014, each depicting a particular customer’s query or concern. Based on our analysis, we identified 9 major categories of concern:

%% See the end of this file:
%% https://docs.google.com/document/d/1LdyNFUBUV0URWayc8kXM2SXaAhWg2FWJO2_8w2IgTvU/edit

\begin{itemize}
\item \textbf{Payment.} This category included tickets filed by the people who had specific problems or general questions regarding paying online bills. Some common problems were error in billing transaction, autopay not correctly set up, inability to understand the online payment system’s GUI etc.
\item \textbf{Subscription Requests.} Thie category included people asking for
service from Further Reach. Others
went further and offered to use their property as new relay
  sites or act as local tech volunteer, in areas like Albion and Gualala. % though locations aren't good
\item \textbf{Quality of Service.} This category included concerns in which people expressed dissatisfaction with the service/plan they were paying for.
The main concern given was that Further Reach seemed to have high costs but slow Internet speed or poor Internet
connectivity. Many users attempted to run SpeedTests to make their point.
\item \textbf{Troubleshooting.} Antenna not installed? System down? Many people call
  about the Internet being down. A large number of reasons ranging from extremely ambiguous to highly specific were given for system outages.
\item \textbf{Questions about Equipment.} In this category, people asked for assistance in set up/installation and maintenance of equipment provided to them by Further Reach in order to help them gain access to the network
\item \textbf{Interference with other Wireless.} There were a few tickets which claimed
that they were receiving poor signal strength due to interference with other
wireless networks within the area, e.g. AT\&T towers.
\item \textbf{Requests for unusual features.} Many tickets asked for availability of various types of Internet services that are available with the download/upload speed the users signed up for such VOIP features, streaming videos using Netflix, paying bills online etc.
\item \textbf{Passwords.} Several tickets provided the SSID/password of their
network and hoped that a technician could solve the connection issue.
\item \textbf{Alternate Pricing.} Tickets in this category asked for details about
Internet plans other than the one the corresponding user had, for example
vacation prices and discounts, non-profits requesting free service, etc.
\end{itemize}

The above categories played a pivotal role in helping us formulate interview questions that mainly target the concerns of the population living outside the Further Reach network regarding high-speed Internet connection.

Essentially, when we analyzed the level of detail provided about the problem
at hand in each ticket, it helped us understand the average level of knowledge
the users had regarding broadband. We also learned about some of common
difficulties users face while accessing various Internet services, one of them
being trouble with paying online bills. Some tickets also helped us understand
the main concerns people have with wireless Internet access. These included
vulnerability to EMF from electronic devices and the health problems
associated with it. Based on the locations of the users filling the tickets,
we were also able to get an insight into their interest in high speed Internet
access. This helped us decide on locations outside the Further Reach network
where we could conduct on-site interviews and try to understand what hold
people back from signing up for high speed Internet access.

\subsection{Interviews with Rural Residents}
\label{sec:interv-with-rural}

With the help of Mendocino Community Network (MCN), we recruited
residents from Mendocino County using a flier that was put on public display
and distributed by Shirley Freriks.
Of the dozen or so residents who expressed interest in the survey, we were able to interview
nine residents, 5 in person and 4 over the phone. The residents were selected among those
who did not currently have access to Further Reach. All of the interviewees
already had Internet access through another provider.

Based on our ticket analysis and interview sessions with Yahel and Zean, we consolidated a list of interview
questions that touched upon the various categories identified in the trouble tickets section. While preparing
the questions, we tried to avoid the assumptions that are our interviewees would have some considerable level
of knowledge about broadband and the services it offers. We also tried to make sure that none of our questions
make the interviewees feel uncomfortable while responding. Table~\ref{tab:questions} lists the questions
we prepared.

We designed our questions to allow the interviewees to express their opinions as openly as possible with as little bias introduced by the
phrasing of the question or our opinions. That said, we
were biased toward the use of technology and may have introduced substantial
biases, by priming their responses in the wording of our questions. In addition
to the interviewee's personal opinion, he or she was also asked about the opinions of her
neighbors.

\begin{table}
\begin{enumerate}
    \item What does broadband mean to you?
    \item Do you have any concerns of high speed? Some part you don’t understand?
    \item Any concerns about health/safety?
    \item In case we put a tower close to your house, what specific concerns u would have?
    \item In case you are provided with high speed Internet connection, what would you mainly use it for?
    \item What is a reasonable price range for high speed Internet in your opinion?
    \item What degree of awareness do you have about the difference between various technology?
    \item Your relatives usage of network and their concerns?
    \item Is there something holding people from signing up for network?
    \item Do you use any specific software to test speed of network?
    \item How many people do you think would be interested in a better broadband service in a 10 mile radius?
    \item How comfortable are you while paying bills online?
    \item Have you considered Internet as substitute for TV~?
    \item Have you contacted Further Reach to setup a network?
    \item What was your reason to participate in the interview?
    \item Do you participate in netbanking or shopping online?
    \item Do you think Internet will be useful for kids
    \item What would you think is most imp that should be included in our educational material? And how would we distribute them?
    \item What is your present Internet provider and how much are you currently paying?
    \item Any negative perception of broadband?
    \item What is your current speed?
    \item What would you think you can do with Internet but you don’t know how to do?
    \item Different people’s experience? Line of sight point of view?
    \item Do you have a sense what the younger generation would do with broadband, which they can’t do now?
\end{enumerate}
\caption{Interview Questions}
\label{tab:questions}
\end{table}

All of the residents that were interviewed expressed an interest in broadband and bringing it
to the Mendocino County. Unfortunately, there were no residents who asked to be interviewed who
were of the opinion that broadband should not be present in Mendocino. In addition, 44.4\% of
the interviewees were considered technically knowledgeable by us.

Aside from the desire to have broadband made available to the homes, the opinions of the
interviewees varied. Many did express some concern for the radiation emitted by antennas
used for wireless Internet access. However, their concerns were not enough to prevent them
from using WiFi at home or from using Further Reach if it were to be made available.
Most interviewed residents were more concerned about the aesthetics of an antenna on his or her
house more than the potential hazard of the RF radiation.
There was a belief that there were those in the community who had major concerns about radiation
from RF antennas. However, these individuals who had major concerns of radiation were described
as being a small minority who had little effect on policy making in the community.

Although many of the residents interviewed described were technically knowledgeable, they felt
that it would be valuable for their community to have technology education made available such
as technology centers for this purpose. Some interviewees also felt that education at the school
for children would be valuable. These same interviewees also described having broadband available
as being most beneficial to the younger population for educational purposes.

There was a technical concern from many residents who felt that wireless broadband by use of
line-of-sight would be infeasible given the terrain that surrounds the house. However, based
on the addresses of these residents and what the residents described, we felt
that it would be technically possible to deliver wireless Internet. However, the cost of
installation was not evaluated.

Cost of installation was a major factor for their choice of Internet access presently. Broadband
through the use of fiber is available to the home. There is currently fiber deployed on many
streets in the area. However, the cost of installation of the fiber has been quoted to be at least \$2000.
Most residents felt this cost was too high if they must also pay for a very high regularly monthly bill, all
estimates exceeding \$100. Residents are currently paying for Internet access between \$50--\$200
through technologies such as satellite or other line-of-sight providers.

Most residents seem to be knowledgeable as to the difference between broadband access and slower speed Internet.
However, there were some confusions that were concerned with some of the details. One often confused topic was
the difference between bandwidth in terms of throughput and a regular bandwidth cap. Another confusion was
as to the bandwidth caps present in cellular services that have no perceived bandwidth cap and the fact that
customers are rate-limited if too much data is used. Lastly, some residents were unclear as to the differences
between the wireless implementations of satellite and terrestrial and its effect on quality of service.

We did a further analysis of the raw interview results, however, due to the small sample size of an area of a small number of people,
we do publish the details here for privacy reasons.


%%% Local Variables:
%%% mode: latex
%%% TeX-master: "main"
%%% End:
