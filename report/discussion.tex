\section{Lessons Learned}
\label{sec:lessons}

As computer scientists who have never been trained in how to conduct qualitative
research, this project was an excellent learning
experience for us. Here are a few lessons we learned along the way:

\begin{itemize}
\item Broadly speaking, qualitative research is really difficult! It is time
consuming and challenging to come up with interview questions that really
focus on the key issues without biasing the responses. Moreover, conducting
interviews is challenging itself, and thoroughly synthesizing the results of
the interviews more-so.
\item It is highly challenging (if not impossible!) to conduct interviews in a
non-biased way. We often walked away from interviews wishing that we could do
them over again, because we felt that the way we had phrased the questions in
a way that primed the interviewee to respond in a particular way. A general
takeaway from this is that we should strive to ask questions in a way that
gets interviewees to do most of the talking, where they can frame issues in
their own words.
\item Snowball sampling seems like a great idea. Sampling bias was pervasive in our interviews; most of our
interviewees were pro-broadband, and we did not manage to find many
anti-broadband viewpoints,
or people who could not afford broadband. This was largely an artifact of the
way that we distributed our request for interviews: having Shirley (MCBA)
email a mailing list of people who are likely actively tuned in to broadband issues.
\item It is hard to draw strong conclusions from our interviews, largely
because of our small sample size. Qualitative research takes a long time! It is
both conceptually difficult and work intensive to come up with strong results
in this area.
\item We have come to believe that our educational materials will be more
useful as a {\em reference} for anyone who comes to Further Reach with
questions, rather than as an vector for convincing many to sign up. From
what we heard in the interviews, there
are numerous reasons why a pamphlet alone is unlikely to succeed in convincing
anyone to sign up: from what we understand, some families in Mendocino
struggle to pay rent, nonetheless
pay for Internet services; some apparently do not want to learn about broadband
and choose to continue living their normal lifestyle (which is of course a
valid choice);
and many more are likely to be indifferent to a simple pamphlet. If we wanted to more effectively
educate the public about broadband services, we would instead (i) hold
in-person information sessions rather than simply handing out a pamphlet, and
(ii) target younger populations who are more likely to be amenable to learning
about technology.
\end{itemize}

\subsection{Future Work / Loose Ends}

Our work currently leaves open several loose ends and directions for future
work. We briefly describe them here.

\begin{itemize}
\item In the next few days, we plan to distribute our educational materials
through the Mendocino Broadband Alliance and through Further Reach's
technicians.
\item As mentioned previously, it would be interesting to analyze how users
engage with the online educational materials. For example, we might add a
quiz at the end to measure educational outcomes, or we might analyze which
sections of the materials receive the most
attention from users.
\item To gain a better understanding of the kinds of issues experienced by
wireless Internet service providers, it might be fruitful to apply a frequency
analysis to the different categories of Further Reach's trouble tickets.
\end{itemize}

%%% Local Variables:
%%% mode: latex
%%% TeX-master: "main"
%%% End:
