\section{Methods}
\label{sec:methods}

Our main research goal is focused around understanding people's perceptions of
broadband services in Mendocino county. Based on what we find from the research,
our second goal is to act on our findings by creating educational materials
designed to address some of the concerns or misconceptions that were raised.

\subsection{Qualitative Research}
\label{subsec:qualitative_research}

We undertook this research project with the broad goal of understanding how
people perceive Internet services, with a specific emphasis on high speed
connection services such as broadband. We wish to analyze how people used
the Internet, and if they had any concerns or misconceptions about Internet
usage, installation, etc.

We conducted this research in several stages. In the initial stages, our goal
was to hone in on a particular topic of inquiry. We then used the results from the
initial stages to design our interview questions, and conduct the interviews
themselves. The stages were as follows:

\begin{enumerate}
\item Analyze support tickets from Further Reach to understand what issues
people commonly have with Internet services.
\item Interview Further Reach technicians (Yahel and Zean) to understand what
topics we should focus on.
\item On-site in-person, and over-the-phone interviews with locals.
\end{enumerate}

\subsection{Educational Materials}

In the second part of our project, our goal was to worked towards analyzing
and addressing these misconceptions through educational materials, that would
aid residents in understanding the working and usage of the Internet in a
better manner. In designing the materials, we strived to include inputs from
interviewees as to what might be valuable to add and how we might best
distribute these materials.

Initially we hoped to measure educational outcomes, i.e. conduct a survey among the people before
distributing materials, then  conduct a similar survey again after distributing the materials
to measure how much they had learned. In the end we decided that this would
not be the most valuable use of time.

%%% Local Variables:
%%% mode: latex
%%% TeX-master: "main"
%%% End:
