\section{Related Work}
\label{sec:related-work}

Two strands of other works are related with our project: i) perceptions and
concerns of Internet technologies; ii) existing education materials. We discuss
each in turn below.

\subsection{(Mis)perceptions and Concerns}
\label{sec:conc-misp}

%% Related Health
The potential safety hazards of human exposure to radio-frequency (RF) waves
have been a major concern, especially with the increasing ubiquity of Wi-Fi
technology. FCC and Wi-Fi Alliance each has dedicated web pages discussing the
health issue \cite{fcc2015radio, wifi2015health}. A range of scientific research
has also been undertaken to understand the impacts and threats of wireless radio
waves to human. A review paper by Foster {\em et.~al}~\cite{foster2013wi}
states that \studyquote{unequivocally, the RF exposures from Wi-Fi and wireless
networks are far below U.S. and international exposure limits for RF energy}
after summarizing the recent research work on Wi-Fi health (as of 2013).

%% health in FR
Further Reach relies heavily on the RF technology in its deployment in Mendocino
County. Different from typical approaches where RF technology is only used in
the last mile of Internet within residents' house, Further Reach also employed
RF to construct the backbone network. Often, the directional attennas are
relatively larger than the typical Wi-Fi devices with smaller omnidirectional
attennas; and the size is causing health concerns to some local residents. Our
project is also tasked to understand such concerns that are unique here and were
not extensively studied before. Note that the directional antennas also bring up
the line-of-sight issue.

%% Other factors
Previous research has identified other concerns that limit the adoption of the
Internet, especially in rural area. LaRose {\em et.~al} suggested the resistance
to constant self-renewal for adopting complex technology, a lack of relevant
content and the affordability of broadband in rural
communities~\cite{larose2007closing}.\footnote{The original paper also has
  discussions of other factors such as technical challenges in rural
  area. Further Reach has solved many such challenges so we focus on the
  remaining factors that local residents have in this paper.} Our work has taken
these factors into consideration during our interview process and the
construction of the education materials.
%% Quotes from larose2007closing:
%% Other limiting factors have been suggested. The adoption of complex
%% technology can be problematic for those living in areas resistant to constant
%% self-renewal (Bell et al., 2004; DeLong, Gahring, Bye, Johnson, & Anderson,
%% 2002). A lack of relevant content (Wilhelm, 2003), low adoption rates in the
%% workplace (Hollifield & Donnermeyer, 2003), and the affordability of
%% broadband in rural communities (Foros & Kind, 2003) might also limit
%% adoption. Because benefits are subjective, people can reject an innovation
%% inconsistent with their norms even if the innovation offers advantages (Kwak,
%% Skoric, Williams, & Poor, 2004).

\subsection{Existing Education Materials}
\label{sec:exist-educ-mater}

Many existing education materials are available online, targeting at audience of
different knowledge levels.

Although the recent massive open online courses (MOOCs) \cite{pappano2012year}
offer excellent materials and instructions, they typically are targeted at
knowledgeable audience (such as college students) rather than the general
public, especially the subject of our research---locals in rural area.

On the other hand, many materials and don't know how to choose.

We summarize a few close related
materials below as references to interested readers. Note that the list is not
meant to be exhaustive.

\noindent {\bf General:} Science education for the general public: \cite{nih, usf}.

{\bf Health:} Organizations, such as Wi-Fi Alliance~\cite{wifi2015health},
EMFacts Consultancy~\cite{emfacts2015wireless}, EMFWise~\cite{emfwise}, have
also created pamphlets or fliers to provide health and safety advice regarding
Wi-Fi.

{\bf Usage:}
\cite{masstech, internetbasics}

{\bf Service Quality:}
\cite{speedmatters, ftth}

Our work is targeted at the specific area followed by our interview results. The
generality is for future work.

%%% Local Variables:
%%% mode: latex
%%% TeX-master: "main"
%%% End:
