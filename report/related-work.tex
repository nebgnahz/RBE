\section{Related Work}
\label{sec:related-work}

Two strands of other works are related with our project: i) perceptions and
concerns of Internet technologies; ii) existing education materials. We discuss
each in turn below.

\subsection{(Mis)perceptions and Concerns}
\label{sec:conc-misp}

%% Related Health
The potential safety hazards of human exposure to radio-frequency (RF) waves
have been a major concern, especially with the increasing ubiquity of Wi-Fi
technology. FCC and Wi-Fi Alliance each has dedicated web pages discussing the
health issue \cite{fcc2015radio, wifi2015health}. A range of scientific research
has also been undertaken to understand the impacts and threats of wireless radio
waves to human. A review paper by Foster {\em et.~al}~\cite{foster2013wi} states
that \studyquote{unequivocally, the RF exposures from Wi-Fi and wireless
  networks are far below U.S. and international exposure limits for RF energy}
after summarizing the recent research work on Wi-Fi health (as of 2013).

%% health in FR
Further Reach relies heavily on the RF technology in its deployment in Mendocino
County. Different from typical approaches where RF technology is only used in
the last mile of Internet within residents' house, Further Reach also employed
RF to construct the backbone network. Often, the directional antennas used by
Further Reach are larger than typical Wi-Fi devices that have smaller
omni-directional antennas. One health concern that locals have is directly
related with the antenna size. Our project is also tasked to understand such
concerns that are unique here and were not extensively studied before. Note that
the directional antennas also bring up the line-of-sight issue.

%% Other factors
Previous research has identified other concerns that limit the adoption of the
Internet, especially in rural area. LaRose {\em et.~al} suggested the resistance
to constant self-renewal for adopting complex technology, a lack of relevant
content and the affordability of broadband in rural
communities~\cite{larose2007closing}.\footnote{The original paper also has
  discussions of other factors such as technical challenges in rural
  area. Further Reach has solved many such challenges so we focus on the
  remaining factors that local residents have in this paper.} Our work has taken
these factors into consideration during our interview process and the
construction of the education materials.
%% Quotes from larose2007closing:
%% Other limiting factors have been suggested. The adoption of complex
%% technology can be problematic for those living in areas resistant to constant
%% self-renewal (Bell et al., 2004; DeLong, Gahring, Bye, Johnson, & Anderson,
%% 2002). A lack of relevant content (Wilhelm, 2003), low adoption rates in the
%% workplace (Hollifield & Donnermeyer, 2003), and the affordability of
%% broadband in rural communities (Foros & Kind, 2003) might also limit
%% adoption. Because benefits are subjective, people can reject an innovation
%% inconsistent with their norms even if the innovation offers advantages (Kwak,
%% Skoric, Williams, & Poor, 2004).

\subsection{Existing Education Materials}
\label{sec:exist-educ-mater}

Many existing education materials are available online, targeting at audience of
different knowledge levels.

On one extreme, the materials from massive open online courses (MOOCs)
\cite{pappano2012year} are quite selected and the instructions are well
prepared. However, current MOOCs are typically targeted at knowledgeable
audience such as college students, rather than the general public, especially
the subject of our research---locals in rural area.

On the other extreme, there are many unorganized resources across the web. Even
with the help of search engine, it will still be a time sink to find the
``right'' materials and understand them, especially for the
non-technologists. Therefore, our work aims to summarize and synthesize these
materials, together with some specific topic targeted at Mendocino County.

Below we offer some resources we have encountered; interested readers can follow
the references or Google more resources. Note that the list is not meant to be
exhaustive.

\begin{itemize}
\item {\bf General:} Organizations such as National Institutes of
Health~\cite{nih}, University of South Florida~\cite{usf} offer resources for
the general public to learn science.

\item {\bf Health:} Many organizations have created pamphlets or fliers to
  provide health and safety advice regarding Wi-Fi. Examples are Wi-Fi
  Alliance~\cite{wifi2015health}, EMFacts
  Consultancy~\cite{emfacts2015wireless}, EMFWise~\cite{emfwise} and so on.

\item {\bf Service Quality:} Speedmatters explained what is high speed Internet
  and why it is necessary~\cite{speedmatters}. Fiber to the Home Council
  explained Fiber as well as the benefit of a high speed.

\item {\bf Usage:} Regarding how Internet can be used, all types of reports and
  tutorials are available. One report from Massachusetts Broadband Institute
  highlights the many ways Internet can help local governance, education,
  digital literacy, health and emergency~\cite{masstech}. Department of
  Communications at Australia has a tutorial about what the Internet can
  do~\cite{internetbasics}.

\end{itemize}

%%% Local Variables:
%%% mode: latex
%%% TeX-master: "main"
%%% End:

%% LocalWords: Wi Fi RF Mis Mendocino al LaRose FR larose
