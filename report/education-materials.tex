\section{Education Materials}
\label{sec:education-materials}

From the interview material, we categorized potential misconceptions into
five key topics. These five topics (below) became the focus of our educational
materials.

\begin{itemize}
\item \textbf{Health.} For this topic we attempt to explain the physical
properties of wireless. For example, we explain how much power is emitted
from antennas as a function of how far away from the antenna you are, and we
distinguish omnidirectional from directional antennas. We explicitly try to
avoid health topics that are still debated within the scientific community,
since it should be up to individuals to take their own stance on these issues,
and we do not wish to impose our personal beliefs on them.
\item \textbf{Line of sight.} Several of the interviewees understood that
wireless requires line of sight, but most (including the interviewers
themselves!) did not have a precise understanding of what is actually required
in practice. We interviewed Yahel to get a better understanding of the exact
requirements of line of sight, and incorporated his explanations into our
educational materials.
\item \textbf{Service quality.} Many interviewees had questions and
potential misconceptions about how to think about and compare
quality of Internet service. We start by explaining basic concepts
(latency, metrics for measuring bandwidth, the difference between bandwidth caps and
bandwidth speeds, etc.), and then try to help them build an intuition for how
to weigh the pros and cons of different services, in terms of concrete examples such
as loading websites of a certain size.
\item \textbf{Cost.} On a related note, we put together a survey of the costs
and service levels provided by competing Internet services in Mendocino.
\item \textbf{Usage.} To address so-called `failures of imagination`, we
surveyed different ways the Internet is used today. These uses include basic
communication (email, chat, video conferencing, social networking), education
(online courses, library materials), finance (online banking, online
shopping), news dissemination, and entertainment.
\end{itemize}

\subsection{Distribution}

We found from our interview with Zean that, because not everyone in Mendocino
feels comfortable using the Internet, we should create our educational
materials in a physical pamphlet form in addition to an online website. We
will distribute the pamphlet through Mendocino Broadband Alliance. We also
plan to make copies of our materials available to Further Reach technicians
who may want to give them to potential subscribers as a reference.

Our online website can be found at \url{http://bit.ly/broadband-education}. The website contains the same
material as the pamphlet, but has a few key advantages over the pamphlet.
Most obviously, online materials might be consumed by a much broader audience. A second
advantage is that we can use the website to indirectly measure how engaged users
are; e.g. we can track how long users spend on the site, as some indication of
whether the materials are pedagogically effective.

%%% Local Variables:
%%% mode: latex
%%% TeX-master: "main"
%%% End:
